\documentclass[10pt,a4paper,final,oneside,notitlepage]{article}
\usepackage[utf8]{inputenc} % Acentos, etc
\usepackage[spanish]{babel} % Idioma

\usepackage[left=2.5cm,right=2.5cm,top=2.5cm,bottom=2.5cm]{geometry} % Margenes
\usepackage{mathptmx} % Times New Roman

\usepackage{multicol} % Dos columnas
\setlength\columnsep{1cm}

\usepackage{titlesec} % Formato de titulos de secciones
\titleformat{\section}{\bfseries\large}{}{0.3em}{}

\pagestyle{empty} % Sin número de página

\makeatletter
% Metadatos
\def\email#1{\gdef\@email{#1}}
\def\@email{\@latex@warning@no@line{No \noexpand\email given}}

\def\universidad#1{\gdef\@universidad{#1}}
\def\@universidad{\@latex@warning@no@line{No \noexpand\universidad given}}

% Título
\newcommand{\titulo} {
	\begin{center}
		{\LARGE\bfseries\@title}
		
		\vspace{2em}
		{\Large\bfseries\@author}
		\vspace{.5em}
		
		{\large\bfseries\itshape\@universidad}
		\vspace{2em}
	\end{center}
}
\makeatother

% Minisección (para las palabras clave, agradecimientos, etc)
\newcommand{\miniseccion}[1] {
	\vspace{1em}
	\noindent\textbf{#1}

	\noindent\ignorespaces
}

% Abstract
\newcommand{\resumen}[1] {
	\miniseccion{Abstract}
	\noindent\textit{#1}
}

% Bibliografia
\newcommand{\bibliografia}[1] {
	\begingroup
	\renewcommand{\section}[2]{}%
	\begin{thebibliography}{9}
	#1
	\end{thebibliography}
	\endgroup
}
\usepackage{amsmath}
\usepackage{amsfonts}
\usepackage{amssymb}

\author{Facundo Javier Gelatti}
\title{Formato para la presentación de Documentos (en \LaTeX)}
\email{javiergelatti@gmail.com}
\universidad{UTN – Fac. Regional Tucumán}

\begin{document}
\titulo

\begin{multicols}{2}
\resumen{``El abstract debería ser considerado como una
miniversión del artículo'' (Day 1991). Describe los
objetivos del estudio, la metodología usada, los
resultados principales del trabajo y sus conclusiones
fundamentales. Un abstract tiene típicamente un único
párrafo y menos de 250 palabras y debe ``permitir a
los lectores identificar el contenido básico del
documento rápida y fielmente, con el fin de determinar
la relevancia del mismo para sus intereses y, por tanto,
para decidir si necesitan leer el documento en su
totalidad'' (definición del American National
Standards Institute). Será escrito en fuente (Times New
Roman, 10, cursiva).}

\miniseccion{Palabras Clave (Times New Roman, 10, negrita)}
Las palabras Claves propuestas por el Autor, para la
indexación del documento. Serán escritas en fuente
(Times New Roman, 10).

\section{Introducción}
La introducción sirve para que los lectores
entiendan el contexto en el que se ha originado el
trabajo y deja claro cuál es el tema básico. Contiene
una descripción clara y precisa del problema que se ha
abordado, explica su relevancia, cita y resume
brevemente los trabajos que definen el problema y
describen soluciones anteriores, para contextualizar la
que se propone. Será escrita en fuente.

\section{Elementos del Trabajo y Metodología}
Es la parte sustancial del trabajo, el lector debe
comprender el método usado con tal detalle que le
permita aplicarlo al mismo o a otro problema. Se
desarrollan los conceptos explicando la metodología
utilizada. Será escrito en fuente.

\section{Resultados}
Los resultados son los que avalarán las conclusiones
y justificarán la utilidad el trabajo realizado.

\section{Discusión}
Explica las relaciones, las tendencias, las posibles
generalizaciones de los resultados observados, sin dejar
de discutir aquellos resultados inesperados que
invaliden total o parcialmente alguna de las hipótesis
iniciales del trabajo. Debe poner los resultados en
relación con los de otros trabajos e indicar las posibles
aplicaciones (o implicaciones teóricas) de los
resultados de la presente investigación.

\section{Conclusión}
La conclusión debería ser la versión condensada de
las secciones anteriores, presentando los resultados
claves encontrados en el trabajo. Debería estar
estrechamente relacionada con los objetivos que fueron
presentados en la introducción. Muchas veces es, junto
con el título, la parte mas leída y por lo tanto debe ser
de comprensión fácil y exacta.

\section{Referencias}
Documentación y bibliografía utilizada. Todas las
publicaciones citadas deberán incluirse en la lista de
referencias. La numeración será secuencial y estará
entre corchetes: [1]. Deberá respetarse el formato de
bibliografía que se detalla a continuación:

\bibliografia{
	\bibitem{asd1}
	A.B. Smith, C.D. Jones, and E.F. Roberts, \emph{``Article Title''}, 
	Journal, Publisher, Location, Date, pp. 1-10.

	\bibitem{asd2}
	Jones, C.D., A.B. Smith, and E.F. Roberts, \emph{Book Title},
	Publisher, Location, Date.
}

\miniseccion{Agradecimientos}
Si existiera, mencionarlos en forma concisa.

\miniseccion{Datos de Contacto}
Nombre y Apellido. Institución. Dirección postal. Email.

\end{multicols}
\end{document}